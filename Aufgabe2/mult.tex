\documentclass[a4paper,12pt]{article}
\usepackage{amsmath}
\usepackage{amssymb}
\usepackage{geometry}
\usepackage{array}

\geometry{margin=2.5cm}

\begin{document}

\section*{Beispiel zur Matrixmultiplikation}

Gewähltes Beispiel: Nehmen wir \( l = 3 \). Dann ist \( m = l - 1 = 2 \) und \( n = l - 1 = 2 \). Matrix \( A \) ist also \( 2 \times 3 \), Matrix \( B \) ist \( 3 \times 2 \), und das Produkt \( C \) wird \( 2 \times 2 \).

Wir definieren:

\[
A = \begin{pmatrix}
2 & 0 & 3 \\
0 & 4 & 0
\end{pmatrix}, \quad
B = \begin{pmatrix}
0 & 5 \\
7 & 0 \\
0 & 1
\end{pmatrix}
\]

Man sieht, \( A \) und \( B \) enthalten mehrere Nullwerte. Nun berechnen wir \( C = A \cdot B \) durch die übliche Summenformel:
\[
c_{ij} = \sum_{k=1}^{l} a_{ik} \cdot b_{kj}
\]

\begin{align*}
c_{0,0} &= A[0,0] \cdot B[0,0] + A[0,1] \cdot B[1,0] + A[0,2] \cdot B[2,0] \\
        &= 2 \cdot 0 + 0 \cdot 7 + 3 \cdot 0 = \mathbf{0} \\[1em]
c_{0,1} &= A[0,0] \cdot B[0,1] + A[0,1] \cdot B[1,1] + A[0,2] \cdot B[2,1] \\
        &= 2 \cdot 5 + 0 \cdot 0 + 3 \cdot 1 = \mathbf{13} \\[1em]
c_{1,0} &= A[1,0] \cdot B[0,0] + A[1,1] \cdot B[1,0] + A[1,2] \cdot B[2,0] \\
        &= 0 \cdot 0 + 4 \cdot 7 + 0 \cdot 0 = \mathbf{28} \\[1em]
c_{1,1} &= A[1,0] \cdot B[0,1] + A[1,1] \cdot B[1,1] + A[1,2] \cdot B[2,1] \\
        &= 0 \cdot 5 + 4 \cdot 0 + 0 \cdot 1 = \mathbf{0}
\end{align*}

Daraus ergibt sich die Resultatsmatrix:
\[
C = A \cdot B =
\begin{pmatrix}
0 & 13 \\
28 & 0
\end{pmatrix}
\]

\end{document}
